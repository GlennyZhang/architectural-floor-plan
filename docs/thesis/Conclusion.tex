\section{Conclusion}
\subsection{Extensions}
In this topic we discuss what additional measures can be taken to improve and extend this program. It will also show what the preparations were to provide an easy implementation for future extensions.

\subsubsection{Object detection}
\label{sub:FutureObjectDetection}
As of now, only doors are being matched with the cascade classifier. There are several objects that could further be identified. Those objects are showers, toilets, baths, windows and different sorts of closets.
There was a try to identify windows with the cascade classifier already. But for the case of windows it is near impossible to find a good classifier. Most of the plans provided have a different structure or a different symbol and even sometimes they are marked differently on one plan. But with the other objects this should be possible. Those objects would help remove noise on the initial plan if you could remove them after they were detected. Additionally these object define different zones as described in section~\ref{sub:ZoneDetection}. Therefore this project already has trained cascade files for each of those objects. Additionally the class that implements the cascade classifier in this project can reference a cascade file and discover the object trained. Therefore all steps to detect the objects are already provided. It is the implementation in a workflow and the usage to define zones that are not yet implemented and need work. In our opinion, implementing this additional detection will make detecting the walls easier as there is less noise and it will improve the room detection as a direct result of that.

\subsubsection{Zone detection}
\label{sub:ZoneDetection}
The zone detection was originally one of the goals of this project. As there was not enough time, we provided a lot of work to make it easily implementable for a future project. As described in the object detection~\ref{sub:FutureObjectDetection}, these zones are defined by different objects.
There are three different zones:
\begin{table}[h]
	\centering
	\label{tab:Zones}
	\begin{tabular}{@{}lll@{}}
		"Blindflächen" & This zone consists out of the elements showers, baths and small \\
		& walls that are an extension to the real wall.\\
		"Randzonen" &  This zone is connected to all windows on the outside of a house. \\
		"Stellflächen" & This zone consists of kitchen combinations and closets.\\
	\end{tabular}
\end{table}	

The reason different zones exist are that there will be different density of heating tubes or none at all under those different zones. The idea is now that combined with the rooms found to split each room into these zones and then additionally return the different zones. The big problem with this addition is, that windows are difficult to detect and therefore the "Randzone" is hard to detect without a heuristic. The current solution to window closing~\ref{sub:WallClosing} could be extended to find empty spaces in between the walls and then define that there is a window. This can be a solution but is prone to difficult structures on the outside wall (like balcony's etc.) and special constructs in the outer walls. Basically any big white space in a wall could possibly result in a gap and would then be recognized as a window. Therefore if there is a possibility to standardize the symbol for a window and then detect it with our cascade classifier. This detection would be a lot smoother.

\subsection{Acknowledgment}
We would like to thank both the Planfabrik GmbH and the Fachhochschule Nordwestschweiz for the possibility to work on such an interesting problem. In specific we would like to thank Oliver and Patrick Stalder for their great support and their insight provided into their work. They both were very supportive and always very interested which made them a pleasure to work with! Additionally we would like to thank Simon Schubiger as our coach for his coaching and all the ideas we were able to discuss. We are very thankful for his continued support and all of his advice!
\todo{Danksagung!}