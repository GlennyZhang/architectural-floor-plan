\section{Theory}
\subsection{Information Segmentation}
\subsubsection{Erosion and Dilation}
In our project we need erosion to strengthen small features that we want to be highlighted as well as to get rid of small details in the picture itself. We use it in early preprocessing. It helps together with the dilation to clean up pictures to further process them without too many interfering objects in the image. The main usage is to extract an image that features all the walls and has as few as possible other lines on it. \cite{burger_burge_2016}
\\
For our project we use the binary grayscale erosion.
The erosion uses the following principle to work.


\[A \ominus B = \{ z\in E | B \subseteq A \}\]  
\[where: B_{z} = \{b+z | b \in B\} with \forall z \in E \]

A is a binary image in the Euclidean space or an integer grid. The erosion is done with the structuring element B on the image A. The structuring element has to be a subset or equal to A, otherwise there will be no erosion. The mask B will be applied to all pixels possible, if the mask fits on the center pixel of A, the value will be retained. Otherwise it will get deleted. This means that only when B is completely contained in A, values of pixels are retained.
\\
Binary dilation uses the exact same process. The only difference compared to the erosion is that the deletion of the pixel will not be setting pixel values to black (0) but to white (1). Both of those processes are inherently the same and can be summed up under the term morphological operation. These are altering the image with a mask, as we did here for erosion and dilation.




\subsubsection{Distance transformation}
The distance transformation is as well as the erosion and dilation a morphological operation. In our project it serves the purpose to find bassins for the watershed algorithm. This is used to define the foreground picture. We will explain this further in the wathershed discussion.
It is used after preprocessing to find the center of what is supposed to be rooms. This algorithm is only used in combination with the watershed.

The algorithm that openCV uses is called cvDistTransform and uses an euclidean distance to calculate the output image. The input for this transformation is a binary image \[I(u,v) = I(x)\] which will be separated into a foreground and background image.
\[FG(I) = {x | I(x) = 1}\]
\[BG(I) = {x | I(x) = 0}\]
The formula for the distance transformation of I, D(x) itself is defined as:
\[D(x) :=\min_{x' \in FG(I)} dist(x,x') \]
It applies to any pixel x = (u,v) of the input image. If the pixel is part of the foreground image FG(I) then the result of the transformation D(x) is zero. As said above, the distance formula dist(x,x') is the euclidean distance (also called L\textsubscript{2}-Norm) between two pixels.
\[dist(x,x') = ||x - x'|| = \sqrt{(u - u')^2 + (v - v')^2}\]

Since the exact calculation with the euclidean distance takes a lot of computational power openCV uses the Chamfer-Algorithm instead. This algorithm runs a two masks over the image. The first mask M\textsubscript{L} is run over in a diagonal manner over the image starting in the top left corner of the image. The second mask M\textsubscript{R} does the same in reverse starting at the bottom right corner. Those masks are a matrix that represent the euclidean distance between the center and the pixels around it. Each mask only changes the pixels in the direction they are run through the image \cite{burger_burge_2016}. As a result, the masks look similar to the following ones:
\[M_{L} = \begin{bmatrix} . & . & .\\ . & x & m_1 \\ m_2 & m_1 & m_2 \end{bmatrix}\]
\[M_{R} = \begin{bmatrix} m2 & m1 & m2 \\ m1 & x & . \\ . & . & . \end{bmatrix}\]

This all leads to the resulting effect shown in the image below.
\todo{Make image of distance transformation before and after}
Source: http://stackoverflow.com/questions/7426136/fastest-available-algorithm-for-distance-transform

\subsubsection{Find contours}
The find-contours algorithm will not receive a detailed description, since it does little more than a simple threshhold in our program. What it is used for is to create the bassins for the watershed algorithm based on the distance transformation. The distance transformation gives us a way to find bassins that are certainly in a room. With the contours-algorithm we create a sharp border for those bassins. Therefore the resulting has a bassin in the center of every room. We can guarantee that this bassin is part of a room and in no way possible part of a wall. In the watershed we will make use of that fact, we will try to connect all bassins to find separate rooms for those that don't connect to others. A more detailed explanation you will find in the watershed section.

\subsubsection{Corner detection}
The corner detection algorithm is used to find corners in an image that contains mostly walls. Its purpose is to find the edges of all walls and therefore provide important information for the door gap closing algorithm. Why this information is needed will be explained in the section about this door gap closing algorithm.
This project uses the harris corner algorithm due to its reliability, simplicity in implementation and the good results we got after the first uses. The algorithm itself is implemented by OpenCv.
The idea of the algorithm is that there are edges wherever the gradient of the image function is high for more than one direction.
First it calculates the basic partial derivatives of the image function I\textsubscript{x}(u,v) in horizontal and vertical direction. This is done for every position in the image (u,v).
\[A(u,v) = I_{x}^2(u,v)\]
\[B(u,v) = I_{y}^2(u,v)\]
\[C(u,v) = I_{x}^2(u,v) * I_{y}^2(u,v)\]

All of those elements will be part of a local structure matrix M.
\[M = \begin{pmatrix} A & C \\ C & B \end{pmatrix}\]
Afterwards all three scalar fields will be smoothed with a linear gaussian filter and be represented as a Matrix \begin{math}\bar{M}\end{math}.
The two eigenvalues of the Matrix \begin{math}\bar{M}\end{math}
\[\lambda_{1,2} = \dfrac{trace(\bar{M})}{2} \pm \sqrt{(\dfrac{trace(\bar{M})}{2})^2 - det(\bar{M})} \]
are positive and contain important information about the local structure. In a area with no corners \begin{math}\bar{M} = 0\end{math}  as well as the two eigenvalues \begin{math}\lambda_1 = \lambda_2 = 0\end{math}. To find a corner there needs to be a strong edge in the main direction (bigger eigenvalue) as well as in the direction of its normal (smaller eigenvalue). 
To find the corner strength itself the harris corner detection uses the function 
\[Q(u,v) := det(\bar{M}(u,v)) - \alpha * (trace(\bar{M}(u,v)))^2\]
whereas the parameter \begin{math}\alpha\end{math} defines the sensitivity of the detector. A candidate for a corner is found when
\[Q(u,v) > t_H\]
with the threshold t\textsubscript{H} being around 10 000 to 1 000 000. To find the exact corner it finds groups of candidates that are close to each other are all eliminated but the best candidate.

\subsubsection{Door gap closing algorithm}

The cascade door training algorithm has found the rectangles that represent the doors. What is needed in the end though is not the door itself but the space inbetween the two walls that the door connects. This is so that we can seperate the different rooms that have doors inbetween.
This algorithm combines the edges that have been found with the Harris-Corner detection and the doors from the cascade door algorithm. It uses a heuristic that works as following. The door algorithm provides four points that mark the edge of the recognized door and build a rectangle.
Added to the rectangle is  a threshold in all directions in which the algorithm searches for edges. The idea is that due to the structure of walls there will be a rectangle made out of four edge points inside the door area that represents the space inbetween the walls. It is expected to only find one rectangle, due to the fact that it makes no sense to have a wall go through a door. As well as the fact that there should be no further lines because it it a cleaned up image and any other line except the walls should not be in that image.
To find the rectangle it connects two edges and calculates the angle compared to the x-Axis for all pairs of edges available. Now all the angles get compared to each other. If two edges AB and CD have a similar angle there will be another comparison for the angles between AC and BD or AD and BC. If one of those two is also of a similar angle we have found our door-rectangle.
\todo{Add image of how those edges are found}

\subsection{Structural Analysis}
\subsubsection{Orientated FAST and Rotated BRIEF}
\todo{Write why SURF not used: License}
\subsubsection{Cascade training}
\subsection{Semantic Analysis}
\subsubsection{Hough transformation}
\subsubsection{Watershed}
The watershed-algorithm in our project is used for segmentation of the different rooms. It can find rooms indifferent of its shape.
\\
The algorithm is processed on a grayscale image on which the color intensity is analogous to the height in a height map. The watershed in use does flooding. The idea is to place a water source in each regional minimum and flood the entire relief. It will stop if it meets a different watersource or an impassable barrier.

The basic algorithm uses three different markings for all regions. The values I for each region can be:
\[I(u,v) = \begin{cases} 0 & background \\  255 & foreground \\ 1..254 & regions \end{cases} \]

It is a really simple algorithm where it starts with looking for an unmarked foreground pixel. All pixels of the region that are connected to this starting pixel will be visited and marked. There are three different ways to do this process. It can be a recursive, an iterative depth-first or breadth-first flood filling. It is not clear what implementation OpenCv uses. In the end all of the methods lead to a similar result and thus will not be discussed further. 

