\section{Theory}
\subsection{Preprocessing}
\subsubsection{Erosion and Dilation}
\subsubsection{Distance transformation}
\subsection{Object detection}
\subsubsection{Orientated FAST and Rotated BRIEF}
\subsubsection{Cascade training}
\subsection{Room detection}
\subsubsection{Hough transformation}
\subsubsection{Watershed}
The watershed-algorithm in our project is used for segmentation of the different rooms. It can find rooms indifferent of its shape.

The algorithm is processed on a grayscale image on which the color intensity is analogous to the height in a heightmap. The watershed in use does flooding. The idea is to place a water source in each regional minimum and flood the entire relief. It will stop if it meets a different watersource or an impassable barrier.
