\section{Annex}

\subsection{Figures}
\begin{table}[h]
\centering
\caption{DWG / DXF evaluation matrix.}
\label{tbl:DWGEvaluationMatrix}
\begin{tabular}{@{}llllllll@{}}
\toprule
Name         & Vendor         & Price        & Last Update      & License           & Read & Write & Comment                             \\ \midrule
YCAD Library & Ed Karlo       & -            & August 07, 2015  & LGPLv2            & Yes  & ?     & Very confusing \& no documentation. \\
Teigha       & ODA            & 2000 USD / Y &                  & Commercial        & Yes  & Yes   &                                     \\
Kabeja       &                & -            & March 12, 2008   & Apache License v2 & Yes  & No    &                                     \\
Tika         & Apache         & -            & October 19, 2016 & Apache License v2 & Yes* & No    & *Meta text reader.                  \\
jnetcad      & Johannes Raida & ?            & April 28, 2016   & Commercial        & Yes* & Yes*  & *Only converter for 3D Objects.     \\
CaffViewer   & DeCaff         & 1350 Euro    & May 17, 2016     & Freeware          & Yes  & Yes   &                                     \\ \bottomrule
\end{tabular}
\end{table}

\pagebreak
\subsection{Developer Guide}
This section describes, how to extend the application developed in this work. It is split into different tasks, which will most likely be done in the future.

\subsubsection{Adding new algorithm}
We recommend, to split up different parts of a new process into different algorithms, which enhances the maintainability of an algorithm.

To add a new algorithm you have to create a new class, which implements the $IAlgorithm$ interface. The interface and algorithms are described in section~\ref{sub:algorithm}, but we will give you here a short overview over the interface.

\begin{figure}[h]
  \centering
      \includegraphics[width=0.6\textwidth]{IAlgorithm_CD}
  \caption{Algorithm interface class diagram.}
  \label{fig:IAlgorithm_CD_DG}
\end{figure}

In figure~\ref{fig:IAlgorithm_CD_DG} you see the methods, which have to be implemented to run the algorithm. The very basic version of an algorithm just returns the input image as it is (Listing \ref{lst:basicAlgorithm}).

\begin{lstlisting}[caption={Basic version of an algorithm.}, label={lst:basicAlgorithm}, language=Kotlin]
class MyAlgorithm : IAlgorithm
{
    override val name: String
        get() = "MyAlgorithm"

    override fun run(image: AFImage, history: MutableList<AFImage>): AFImage {
        return image
    }
}
\end{lstlisting}

\subsubsection{Extending the user interface}
\subsubsection{Training new classifiers}