\section{Abstract}
The goal of this work is, to do a fast and robust room detection on floor plans. The idea is, that a wide range of non standardized floor plans can be analyzed time efficient with little drawbacks in its precision.
The used workflow consists of several algorithms, that are combined to deliver the expected result. It consists of \textit{Morphological cleaning} for noise removal, \textit{Machine Learning} and \textit{Convex Hull closing} for gap closing and a \textit{Connected Component analysis} for the final room detection. It is the best result out of different approaches that were tested. All of the approaches are discussed in greater detail in the "Implementation" part of this work. All of the algorithms used use an image of a plan  as the start for detection and return the location and size of each room as a CSV-table or SVG-vectors. The project prepares all data to return the rooms as a dwg- or dxf-Format for a CAD-Program, but the license for a library to convert the format is out of budget.
The algorithm implemented shows improvement in room detection accuracy compared to similar works done in the last few years. The room accuracy for the algorithm itself for cleaned images is on average 84 percent. It can be improved with user interaction by using the editor. The room detection rate goes up to a 100 percent. Generally, using the algorithm for room detection improves the clicks and time used compared to the program used at the PlanfabrikGmbH considerably.