\section{Abstract}
The goal of this work is, to do a fast and robust room detection on floor plans. The idea is, that a wide range of non standardized floor plans can be analyzed time efficient with little drawbacks in its precision.
The used workflow consists of several algorithms, that are combined to deliver the expected result. It consists of \textit{Morphological cleaning} for noise removal, \textit{Machine Learning} and \textit{Convex Hull closing} for gap closing and a \textit{Connected Component analysis} for the final room detection. It is the best result out of different approaches that were tested. All of the approaches are discussed in greater detail in the "Implementation" part of this work. All of the algorithms used use an image of a plan  as the start for detection and return the location and size of each room as a CSV-table or SVG-vectors. The project prepares all data to return the rooms as a dwg- or dxf-Format for a CAD-Program, but the license for a library to convert the format is out of budget.

\todo{Abstract mehr in der Form: Ausgangslage, Vorgehen, Resultat}
This work's main topic is the analysation of floor plans to detect and mark areas where different types of floor heatings will be installed. The idea is to automate this process for simple floor plans and provide additional tools to correct possible errors in more complicated plans.
This work will be split into several parts. The first part will describe several algorithms used for preprocessing the image to remove any noise. Additionally, the algorithms for room detection and object detection will be explained.
The second part will contain the whole project management and implementation. This includes for example a timetable of our work, the class structure and much more information concerning the project setup and execution.
Last but not least, we will present and discuss the results of our work. This also includes features that may be built in the future and the limitations of our work.