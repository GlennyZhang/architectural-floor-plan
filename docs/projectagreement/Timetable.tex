\chapter{Timetable}

This chapter shows our rough planning of the project. There are likely to be changes to the length of certain activities that might take longer or shorter than displayed. What should not change is the date and content of the milestones, which will be described below.

\section{Milestone 1}
The focus in milestone one is on the project setup and planning. The idea is to have a project agreement signed and the software architecture done. Part of the project agreement will be to define a way to measure the success with one or several metrics.

\section{Milestone 2}
Milestone is all about searching suiting algorithms, constructing the software and then testing it. At the end of this milestone we expect to be able to:

\begin{itemize}
	\item Reading DWG/FXF or images as input.
	\item Room polygon recognition in floor plans.
	\item Room polygon export for calculation software.
\end{itemize}

\section{Milestone 3}
At the end of this milestone the software should be completed, aside from little improvements to be made. The following additional points are to be expected: 
\begin{itemize}
	\item Manual edit possibility of recognised rooms.
	\item Confidence measurement and warning system.
	\item Basic zone recognition.
	\item Faster work-process (measurement by metric in section \ref{sec:metric})
\end{itemize}

\section{Transition}
At the end of the project there is a three week transition phase. It is used to deliver the software and in case errors pop up to fix them. It might also be used to finish additional features but not to start to create new ones.